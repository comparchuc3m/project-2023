\subsubsection{Colisiones de partículas}

En esta etapa, se tienen en cuenta las colisiones de las partículas con las
paredes del recinto para actualizar el vector aceleración.
Dado que los índices para un bloque de la malla son
$c_x$, $c_y$ y $c_z$, se tienen en cuenta las siguientes opciones:

\begin{itemize}
\item $c_x=0$
\item $c_x=n_x-1$
\item $c_y=0$
\item $c_y=n_y-1$
\item $c_z=0$
\item $c_z=n_z-1$
\end{itemize}

\subparagraph{Colisiones con los límites del eje x}

Cuando se fija $c_x$, para cualquier combinación de $c_y$ y $c_z$ se
procesan todas las partículas. Primero, se calcula la coordenada $x$ de la 
nueva posición.

\[
x = \vec{p}_x + \overrightarrow{hv}_x \cdot \Delta t \\
\]

Después, la diferencia con el límite se compara con el \textmark{tamaño de la
partícula} ($d_p$).

\[
\Delta x = 
\begin{cases}
d_p - (x - x_{min}) & \text{if } c_x = 0\\
d_p - (x_{max} - x) & \text{if } c_x = n_x -1\\
\end{cases}
\]

Si $\Delta x$ es mayor que $10^{-10}$, se actualiza la aceleración,
teniendo en cuenta la \textmark{colisiones de rigidez} ($c_s$) y el
\textmark{amortiguamiento} ($d_v$).

\[
a_x = 
\begin{cases}
  a_x + c_s \cdot \Delta x - d_v \cdot v_x & \text{if  } c_x = 0\\
  a_x - c_s \cdot \Delta x + d_v \cdot v_x & \text{if  } c_x = n_x-1\\
\end{cases}
\]

\subparagraph{Colisiones con los límites del eje y}

Cuando se fija $c_y$, para cualquier combinación de $c_x$ y $c_z$ se
procesan todas las partículas. Primero, se calcula la coordenada $y$ de la 
nueva posición.

\[
y = \vec{p}_y + \overrightarrow{hv}_y \cdot \Delta t \\
\]

Después, la diferencia con el límite se compara con el \textmark{tamaño de la
partícula} ($d_p$).

\[
\Delta y = 
\begin{cases}
d_p - (y - y_{min}) & \text{if } c_y = 0\\
d_p - (y_{max} - y) & \text{if } c_y = n_y -1\\
\end{cases}
\]

Si $\Delta y$ es mayor que $10^{-10}$, se actualiza la aceleración,
teniendo en cuenta la \textmark{colisiones de rigidez} ($c_s$) y el
\textmark{amortiguamiento} ($d_v$).

\[
a_y = 
\begin{cases}
  a_y + c_s \cdot \Delta y - d_v \cdot v_y & \text{if  } c_y = 0\\
  a_y - c_s \cdot \Delta y + d_v \cdot v_y & \text{if  } c_y = n_y-1\\
\end{cases}
\]

\subparagraph{Colisiones con los límites del eje z}

Cuando se fija $c_z$, para cualquier combinación de $c_x$ y $c_y$ se
procesan todas las partículas. Primero, se calcula la coordenada $z$ de la 
nueva posición.

\[
z = \vec{p}_z + \overrightarrow{hv}_z \cdot \Delta t \\
\]

Después, la diferencia con el límite se compara con el \textmark{tamaño de la
partícula} ($d_p$).

\[
\Delta z = 
\begin{cases}
d_p - (z - z_{min}) & \text{if } c_z = 0\\
d_p - (z_{max} - z) & \text{if } c_z = n_z -1\\
\end{cases}
\]

Si $\Delta z$ es mayor que $10^{-10}$, se actualiza la aceleración,
teniendo en cuenta la \textmark{colisiones de rigidez} ($c_s$) y el
\textmark{amortiguamiento} ($d_v$).

\[
a_z = 
\begin{cases}
  a_z + c_s \cdot \Delta z - d_v \cdot v_z & \text{if  } c_z = 0\\
  a_z - c_s \cdot \Delta z + d_v \cdot v_z & \text{if  } c_z = n_z-1\\
\end{cases}
\]
