\subsection{Iniciación de la simulación}

Durante la iniciación, se abre el archivo de entrada y se leen los parámetros de
simulación principales (partículas por metro y número de partículas).
Después, se calculan otros parámetros que dependen de éstos.

Those parameters include:
\begin{itemize}
\item La \textmark{longitud de suavizado} ($h$).
\item La \textmark{masa de la partícula} ($m$).
\item El \textmark{vector de tamaño de malla} ($\vec{n} = (n_x, n_y, n_z)$).
\item El \textmark{tamaño de bloque de malla} ($\vec{s} = (s_x, s_y, s_z)$).
\end{itemize}

La información para cada partícula se lee del archivo de entrada. Aunque los
archivos de entrada utiliza simple precisión para los valores en coma flotante,
tan pronto como los valores se leen se convertirán a doble precisión. Las
partículas reciben un identificador numérico entero a partir del \cppid{0}.

Una vez se ha leído una partícula, se determina su correspondiente bloque en la
malla y la partícula se añade al conjunto de partículas de ese bloque.

Si el número de partículas encontrado en el archivo no coincide con el parámetro
\textmark{número de partículas} (\cppid{np}), se genera un mensaje de error y
el programa termina con el código de error
\cppid{-5}. Por ejemplo, si la cabecera tiene el valor \cppid{4750} para
\cppid{np} pero el archivo contiene 4700 partículas, el siguiente mensaje se
envía a la salida estándar de errores.

\begin{lstlisting}[style=terminal,escapechar=@]
@\$@ aos 2000 init.fld final.fld
Error: Number of particles mismatch. Header: 4750, Found: 4700.
\end{lstlisting}

Si el archivo se lee con éxito, los valores de los parámetros se envían a la
salida estándar.

\begin{lstlisting}[style=terminal,escapechar=@]
@\$@ aos 2000 init.fld final.fld
Number of particles: 4800
Particles per meter: 204
Smoothing length: 0.00830882
Particle mass: 0.00011779
Grid size: 15 x 21 x 15
Number of blocks: 4725
Block size: 0.00866667 x 0.00857143 x 0.00866667
\end{lstlisting}

