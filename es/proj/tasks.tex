\section{Tareas}

\subsection{Desarrollo de software}

\subsubsection{Programa a desarrollar}

Esta tarea consiste en el desarrollo de una versión secuencia de la aplicación
descrita utilizando C++20. Por favor, ten en cuenta que no se permite
utilización de varios hilos de ejecución (\emph{multithreading}) ni de
paralelismo de ninguna clase.

Para el desarrollo del programa los equipos pueden considerar distintas
alternativas como el uso de estructuras de arrays, arrays de estructuras o bien
alternativas híbridas entre ambos enfoques.

\subsubsection{Componentes}

Se desarrollarán los siguientes componentes

\begin{itemize}
  \item \cppid{sim}: 
        Biblioteca con todos los componentes usados desde el programa principal.

  \item \cppid{utest}: 
        Pruebas unitarias de todos los componentes de la biblioteca \cppid{sim}.

  \item \cppid{ftest}:
        Pruebas funcionales de la aplicación.

  \item \cppid{fluid}: 
        Programa ejecutable con la aplicación.

\end{itemize}

A continuación, se describen algunos componentes con más detalle:

\paragraph{sim}: Componentes de la biblioteca de simulación.

Contendrá, al menos, los siguientes archivos fuente:
\begin{itemize}
  \item \cppid{progargs.hpp, progargs.cpp}: 
        Tratamiento de argumentos para los parámetros de \cppid{main()}.
  \item \cppid{grid.hpp, grid.cpp}: 
        Representación de la malla.
  \item \cppid{block.hpp, block.cpp}: 
        Representación de un bloque.
\end{itemize}

\paragraph{utest}: Pruebas unitarias.

Contendrá, al menos, los siguientes archivos fuente:
\begin{itemize}
  \item \cppid{progargs\_test.cpp}: 
        Pruebas unitarias para \cppid{progargs}.
  \item \cppid{grid\_test.cpp}:
        Pruebas unitarias para \cppid{grid}.
  \item \cppid{block\_test.cpp}:
        Pruebas unitarias para \cppid{block}.
\end{itemize}

\paragraph{fluid}: Programa con la aplicación.

Contendrá un único archivo fuente:
\begin{itemize}
  \item \cppid{fluid.cpp}: 
        Contendrá solamente una función \cppid{main()} para la aplicación.
        No incluirá ninguna función adicional.
\end{itemize}

\subsubsection{Compilación del proyecto}

Todos los archivos fuente deben compilar sin problemas y no emitirán ninguna
advertencia del compilador. Se incluirá un archivo de configuración de CMake
con las siguientes opciones:

\begin{lstlisting}[title={Main CmakeLists.txt},frame=single]
cmake_minimum_required(VERSION 3.26)
project(fluid LANGUAGES CXX)

set(CMAKE_CXX_STANDARD 20)
set(CMAKE_CXX_STANDARD_REQUIRED ON)
set(CMAKE_CXX_EXTENSIONS_OFF)
add_compile_options(-Wall -Wextra -Werror -pedantic -pedantic-errors)

set(CMAKE_CXX_FLAGS_RELEASE "-march=native")

add_subdirectory(sim)
add_subdirectory(fluid)

enable_testing()
add_subdirectory(utest)
add_subdirectory(ftest)
\end{lstlisting}

El archivo de configuración de CMake para el directorio \cppid{sim} incluirá las
siguientes opciones:

\begin{lstlisting}[title={Library sim CmakeLists.txt},frame=single]
add_library(sim file1.cpp file2.cpp file3.cpp)
target_include_directories(sim PUBLIC ..)
\end{lstlisting}

El archivo de configuración de CMake para el directorio \cppid{fluid} incluirá las
siguientes opciones:

\begin{lstlisting}[title={Program fluid CmakeLists.txt},frame=single]
add_executable(fluid fluid.cpp)
target_link_libraries(fluid sim)
target_include_directories(fluid PUBLIC ../common)
\end{lstlisting}

Ten en cuenta que estas reglas son solamente ejemplos.

Recuerda que todas las evaluaciones se realizarán con las optimizaciones del
compilador activadas con la opción de CMake
\cppid{-DCMAKE\_BUILD\_TYPE=Release}.

\textbad{IMPORTANTE}: 
La única biblioteca permitida es la biblioteca estándar de C++.
No se permite bibliotecas externas.

\textbad{EXCEPCIÓN}: 
Se permite el uso de la biblioteca de soporte a las \emph{C++ Core Guidelines}.
La última versión se puede obtener en:
\url{https://github.com/microsoft/GSL}.

\subsubsection{Reglas de calidad de código}

El código fuente debe estar bien estructurado y organizado, así como
apropiadamente documentado.
Se recomienda (aunque no se requiere) seguir las \textemph{C++ Core Guidelines}
(\url{http://isocpp.github.io/CppCoreGuidelines/CppCoreGuidelines}).

Se publicará un conjunto de reglas en un documento separado.

\subsubsection{Pruebas unitarias}

Se definirá un conjunto de pruebas unitarias que también se entregarán.
Se recomienda, aunque no se requiere, el uso de GoogleTest
(\url{https://github.com/google/googletest}).

En cualquier caso, se entregarán evidencias de que hay pruebas unitarias
suficientes.

\subsubsection{Uso de herramientas de IA}

El uso de herramientas basadas en IA durante el proyecto está permitido.
No obstante, se debe tener en cuenta lo siguiente:

\begin{itemize}

\item Si usas una herramienta basada en IA, se deben declarar los usos en 
la sección de diseño de la memoria del proyecto. No se realizará ninguna
penalización en la declaración siempre que se incluya una declaración de uso.

\item No se dará soporte al uso de dichas herramientas. En particular,
si tienes preguntas sobre tu código, deberás ser capaz de explicar dicho
código.

\item Ten en cuenta que algunas herramientas de IA pueden generar código
inseguro o código con bajo rendimiento.

\item Cualquiera de tus profesores podrá requerirte una explicación de tu propio 
código.

\end{itemize}

\subsection{Evaluación del rendimiento y la energía}

Esta tarea consiste en la realización de una evaluación del rendimiento.
Para llevar a cabo la evaluación del rendimiento, se medirá el tiempo total
de ejecución. Además, también se medirá la energía y se derivará la potencia.

Todas las evaluaciones del rendimiento se realizarán en un nodo del clúster
\cppid{avignon}.

Representa en una gráfica los tiempos totales de ejecución, uso de energía y
potencia para distintos números de iteraciones.

\textbf{La memoria del proyecto incluirá conclusiones derivadas de los
resultados}.
Por favor, no te limites a una mera descripción de los datos.
Trata de encontrar explicaciones convincentes de estos resultados.

