\section{Calificación}

Las calificaciones finales para este proyecto se obtienen de la siguiente forma:

\begin{itemize}
  \item Rendimiento: 15\%.
  \item Energía usada: 15\%.
  \item Pruebas unitarias: 10\%.
  \item Pruebas funcionales: 5\%.
  \item Calidad del diseño: 5\%.
  \item Calidad del código: 10\%.
  \item Evaluación de rendimiento y energía en la memoria: 15\%.
  \item Contribuciones de cada miembro: 20\%.
  \item Conclusiones: 5\%.  
\end{itemize}

\textgood{ADVERTENCIAS IMPORTANTES}:

\begin{itemize}
  \item Si el código entregado no compila, la nota final de la práctica será de 0. 
  \item Si se ignora injustificadamente alguna norma de calidad de código,
        la nota final de la práctica será de 0.
  \item En caso de copia todos los grupos implicados obtendrán una nota de 0.
        Además, se notificará a la dirección de la EPS para la correspondiente
        apertura de expediente disciplinario.
\end{itemize}

\section{Procedimiento de entrega}

La entrega del proyecto se realizará a través de Aula Global.

Para ello se habilitarán 2 entregadores separados:

\begin{itemize}

\item \textgood{Entregador de memoria}. Contendrá la memoria del proyecto, que será
un archivo en formato pdf con el nombre \textmark{report.pdf}.

\item \textgood{Entregador de código}: Contendrá todo el código fuente
necesario para compilar la aplicación.
\begin{itemize}
  \item Debe ser un archivo comprimido (formato zip) con el nombre
        \textmark{fluid.zip}.
\end{itemize}

\end{itemize}

La memoria no deberá exceder de 15 páginas con una fuente mínima de 10 puntos
incluyendo la portada y todas las secciones.  no se tendrá en cuenta en la
corrección los contenidos a partir de la página 16 si fuese el caso. Deberá
contener, al menos, las siguientes secciones: 

\begin{itemize}

\item \textmark{Página de título}: contendrá los siguientes datos:
  \begin{itemize}
    \item Nombre de la práctica.
    \item Nombre del grupo reducido en el que están matriculados los estudiantes.
    \item Número de equipo asignado.
    \item Nombre y nia de todos los autores.
  \end{itemize}

\item \textmark{Diseño original}. 
      Debe incluir el diseño de cada uno de los componentes.
      En esta sección se deben explicar y justificar cada una de las principales 
      decisiones de diseño tomadas.

\item \textmark{Optimización}.
      Debe contener una discusión de las optimizaciones
      aplicadas y su impacto. En particular, deberán indicarse optimizaciones
      realizadas sobre el código fuente original, así como optimizaciones
      activadas con flags de compilación adicionales que se estime oportuno.

\item \textmark{Pruebas realizadas}: 
      Descripción del plan de pruebas realizadas para asegurar la correcta ejecución.
      Debe incluir pruebas unitarias, así como pruebas funcionales de sistema.

\item \textmark{Evaluación de rendimiento y energía}: 
      Deberá incluir las evaluaciones de rendimiento y energía llevadas a cabo.

\item \textmark{Organización del trabajo}:
      Deberá describir la organización del trabajo entre los miembros del equipo
      haciendo explícita las tareas llevadas a cabo por cada persona.
        \begin{itemize}
          \item Debe contener una división del proyecto en tareas.
          \item Las tareas deben ser suficientemente pequeñas como para que se
                pueda asignar una única persona a una tarea.
          \item Todos los integrantes del equipo deberán realizar contribuciones
                relevantes en el diseño y construcción de componentes software.
          \item No se podrá asignar más de una persona a una tarea.
                En tal caso, la tarea debe subdividirse en subtareas.
          \item Se debe indicar el tiempo dedicado por cada persona a cada tarea.
        \end{itemize}

\item \textmark{Conclusiones}.
      Se valorará especialmente las derivadas de los resultados de la evaluación
      del rendimiento, así como las que relacionen el trabajo realizado con el contenido
      de la asignatura.

\end{itemize}
