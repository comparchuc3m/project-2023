\subsection{Análisis de argumentos del programa}

Cada aplicación (\cppid{soa} y
\cppid{aos}) tomará exactamente tres parámetros:

\begin{itemize}
\item \textmark{nts}: 
Número de \emph{time steps} (pasos de tiempo). Este parámetro especifica
durante cuántos paso de tiempo debe ejecutarse la simulación.
\item \textmark{inputfile}: 
Archivo de entrada con el estado inicial de la simulación.
\item \textmark{outputfile}: 
Archivo de salida con el estado final de la simulación.
\end{itemize}

Por ejemplo, el mandato:

\begin{lstlisting}[style=terminal,escapechar=@]
@\$@ aos 2000 init.fld final.fld
\end{lstlisting}

Carga el archivo \cppid{init.fld} ejecuta \cppid{2000} pasos de tiempo y genera
un archivo de salida llamado \cppid{final.fld}.

Si el número de argumentos no es exactamente tres argumentos, 
se generará un mensaje de error y el programa terminará con el código de error 
\cppid{-1}.

\begin{lstlisting}[style=terminal,escapechar=@]
@\$@ aos 
Error: Invalid number of arguments: 0.
@\$@ aos 4
Error: Invalid number of arguments: 1.
@\$@ aos 2000 init.fld
Error: Invalid number of arguments: 2.
@\$@ aos 2000 init.fld final.fld 45
Error: Invalid number of arguments: 4.
\end{lstlisting}

Si el primer argumento no es un número entero,
se generará un mensaje de error y el programa terminará con el código de error 
\cppid{-1}.

\begin{lstlisting}[style=terminal,escapechar=@]
@\$@ aos hello init.fld final.fld
Error: time steps must be numeric.
\end{lstlisting}

Si el número de pasos de tiempo es un número negativo,
se generará un mensaje de error y el programa terminará con el código de error 
\cppid{-2}.

\begin{lstlisting}[style=terminal,escapechar=@]
@\$@ aos -3 init.fld final.fld
Error: Invalid number of time steps.
\end{lstlisting}

Si el archivo de entrada no se puede abrir para lectura,
se generará un mensaje de error y el programa terminará con el código de error 
\cppid{-3}.

\begin{lstlisting}[style=terminal,escapechar=@]
@\$@ soa 1 init.fld final.fld
Error: Cannot open init.fld for reading
\end{lstlisting}

Si el archivo de salida no se puede abrir para escritura, 
se generará un mensaje de error y el programa terminará con el código de error 
\cppid{-4}.

\begin{lstlisting}[style=terminal,escapechar=@]
@\$@ soa 5 init.fld final.fld
Error: Cannot open final.fld for writing
\end{lstlisting}
