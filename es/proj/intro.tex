\section{Introducción}

La simulación de flujos de fluidos se puede lograr con diversos métodos. Uno de
ellos es la simulación de hidrodinámica de partículas suavizada (\emph{Smoothed
Particle Hydrodynamics} o \emph{SPH}). La simulación SPH resuelve las ecuaciones
de Navier-Stokes en pasos de tiempo. Un fluido se modela como una colección de
partículas. Estas partículas tienen interacciones que tienen impacto en su
aceleración, posición y velocidad.

Solamente las partículas que están a una corta distancia tienen una interacción
real. Para acelerar la simulación, las interacciones de larga distancia se
descartan. Para lograr esto, el espacio 3D se divide en una malla de bloques y
solamente se tienen en cuenta las partículas en bloques vecinos.

La simulación incluye varios pasos:
\begin{itemize}
  \item Lectura del estado inicial de la simulación.
  \item Simulación de las partículas para cada paso de tiempo o iteración:
    \begin{itemize}
      \item Reposicionamiento de cada partícula en la malla.
      \item Cálculo de las fuerzas y aceleraciones de cada partícula.
      \item Procesamiento de las colisiones de partículas con los límites del
recinto.
      \item Movimiento de partículas.
      \item Reprocesamiento de los límites del recinto.
    \end{itemize}
  \item Almacenamiento del estado final de la simulación.
\end{itemize}
