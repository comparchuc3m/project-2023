\subsection{La malla de simulación}

Para evita realizar demasiados cálculos, el recinto de simulación completo se
divide en bloques de menor tamaño. Ten en cuenta que el recinto de simulación es
un cuboide rectangular con longitudes diferentes en cada una de las tres
dimensiones.

Para determinar el número de bloques de cada dimensión se divide la longitud en
esa dimensión por el parámetro $h$.

\[
\begin{split}
n_x = \lfloor \frac{x_{max}- x_{min}}{h} \rfloor\\
n_y = \lfloor \frac{y_{max}- y_{min}}{h} \rfloor\\
n_z = \lfloor \frac{z_{max}- z_{min}}{h} \rfloor
\end{split}
\]

El tamaño de un bloque de la malla $\vec{s}$ ($s_x, s_y, s_z$) viene dado por
\[
\begin{split}
s_x = \frac{x_{max} - x_{min}}{n_x}\\
s_y = \frac{y_{max} - y_{min}}{n_y}\\
s_z = \frac{z_{max} - z_{min}}{n_z}\\
\end{split}
\] 

Dada una partícula con vector de posición $\vec{p}$, los índices de bloque se
calcula teniendo en cuenta su distancia en cada dimensión a la coordenada mínima
del recinto y el tamaño de un bloque de la malla ($\vec{s}$).

\[
\begin{split}
i = \lfloor \frac{p_x - x_{min}}{s_x} \rfloor\\
j = \lfloor \frac{p_y - y_{min}}{s_y} \rfloor\\
k = \lfloor \frac{p_z - z_{min}}{s_z} \rfloor\\
\end{split}
\]

Ten en cuenta que para cada dimensión el correspondiente índice no puede ser
menor que $0$ o mayor o igual que el correspondiente número de bloques en esa
dimensión.

\[
\begin{split}
i \in [0, n_x-1]\\
j \in [0, n_y-1]\\
k \in [0, n_z-1]\\
\end{split}
\]

Si un índice está fuera de su rango correspondiente, se debe ajustar al
correspondiente límite.

El papel clave de los bloques de simulación es que solamente usarán para los
cálculos las partículas en el mismo bloque o en bloques contiguos. En general,
para cualquier bloque hay 26 bloques contiguos. Este número puede ser menor
cuando el bloque está en los límites de la malla. Por ejemplo, cuando $i=j=k=0$,
el número de 7.
