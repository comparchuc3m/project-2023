\clearpage
\subsection{Constantes de la simulación}

La tabla~\ref{tab:constants} muestra las constantes utilizadas en la 
simulación:

\begin{table}[hbt!]

\begin{center}
\begin{tabular}{|l|l|r|}

\hline
\textbf{Constante} & \textbf{Descripción} & \textbf{Valor}\\
\hline
\hline

r & \textgood{Multiplicador de radio} & 1.695\\
\hline

$\rho$ & \textgood{Densidad de fluido} & $10^3$\\
\hline

$p_s$ & \textgood{Presión de rigidez} & 3.0\\
\hline 

$s_c$ & \textgood{Colisiones de rigidez} & $3 \cdot 10^4$\\
\hline

$d_v$ & \textgood{Amortiguamiento} & 128.0\\
\hline

$\mu$ & \textgood{Viscosidad} & 0.4\\
\hline

$d_p$ & \textgood{Tamaño de partícula} & $2 \cdot 10^{-4}$\\
\hline

$\Delta t $& \textgood{Paso de tiempo} & $10^{-3}$\\
\hline

\end{tabular}
\end{center}

\caption{Constantes escalares de la simulación.}
\label{tab:constants}
\end{table}

Además, la tabla~\ref{tab:vec-constants} presenta la constantes vectoriales a
utilizar en la simulación. Ten en cuenta que la \textgood{aceleración externa}
($\vec{g}$) es la aceleración de la gravedad, que en este ejemplo actúa a lo
largo del eje y.  El \textgood{límite superior del recinto} ($\vec{b}_{max}$) y
\textgood{límite inferior del recinto} ($\vec{b}_{min}$) define dos vértices
opuestos en un cuboide rectangular.  Este es el dominio cerrado en el que ocurre
la simulación.  $\vec{b}_{max}$ viene dado por el vector $(x_{max}, y_{max},
z_{max})$ y  $\vec{b}_{min}$ viene dado por el vector $(x_{min}, y_{min},
z_{min})$.

\begin{table}[h]

\begin{center}
\begin{tabular}{|l|l|r|r|r|}

\hline
\textbf{Constante} & \textbf{Descripción} & \cppid{x} & \cppid{y} & \cppid{z}\\
\hline
\hline

$\vec{g}$ & \textgood{Aceleración externa} & 0.0 & 9.8 & 0.0\\
\hline

$\vec{b}_{min}$ & \textgood{Límite superior de recinto} & 
\cppid{$x_{min}$} = 0.065 & \cppid{$y_{min}$} = 0.1 & \cppid{$z_{min}$} = 0.065\\
\hline

$\vec{b}_{max}$ & \textgood{Límite inferior de recinto} & 
\cppid{$x_{max}$} = -0.065 & \cppid{$y_{max}$} = -0.08 & \cppid{$z_{max}$} = -0.065\\
\hline 

\end{tabular}
\end{center}

\caption{Constantes vectoriales de simulación.}
\label{tab:vec-constants}
\end{table}

La figura~\ref{fig:cube} muestra una posible mall dividida en bloques
y los límites del recinto $b_{min}$ y $b_{max}$.

\begin{figure}[htb!]
\begin{center}
\input{int/cube.tkz}
\end{center}
\caption{Cube dimension.}
\label{fig:cube}
\end{figure}



Ten en cuenta que pueden ser necesarias algunas constantes matemáticas (p.ej.
$\pi$). En tal caso el valor preferido será el suministrado por la biblioteca
estándar de C++20. Por favor, consulta el espacio de nombres
\cppid{std::numbers} para dichas constantes. 
