\subsubsection{Cálculo de aceleraciones}

Se realizan los siguientes pasos:

\begin{enumerate}
\item Iniciación de aceleraciones.
\item Incremento de densidades.
\item Transformación de densidades.
\item Transferencia de aceleraciones.
\end{enumerate}

\paragraph{Iniciación de densidades y aceleraciones}
Durante este paso se da un valor inicial a la densidad ($\rho_i$)
y el vector aceleración ($\vec{a_i}$) para cada partícula.

La densidad se inicia a \cppid{0}.
\[
\rho_i = 0
\]

El vector aceleración se inicia al valor de la \textgood{aceleración externa}.
\[
\vec{a}_i = \vec{g}
\]

\paragraph{Incremento de densidades}
En este paso, para cada bloque de la malla de simulación, se consideran todos 
los bloques contiguos así como el bloque actual. Para cada par de partículas
$i$ y $j$, se incrementan las densidades.

\[
\Delta \rho_{ij} = 
    \begin{cases}
      (h^2 - \|\vec{p_i} - \vec{p_j}\|^2)^3 & 
          \text{si  } \|\vec{p_i} - \vec{p_j}\|^2 < h^2\\
      0 & 
          \text{si  } \|\vec{p_i} - \vec{p_j}\|^2 \geq h^2\\
    \end{cases}
\]

Ten en cuenta que el incremento de densidad $\Delta \rho_{ij}$ se aplica a ambas
partículas: $i$ y $j$.

\[
\begin{split}
\rho_i = \rho_i + \Delta \rho_{ij}\\
\rho_j = \rho_j + \Delta \rho_{ij}\\
\end{split}
\]

No obstante, ten especial cuidado para evitar que una partícula se incremente dos
veces con el mismo incremento (ya que $\Delta \rho_{ij} = \Delta \rho_{ji}$).

\paragraph{Transformación de densidad}
Para cada densidad calculada, se realiza una transformación lineal.

\[
\rho_i = (\rho_i + h^6) \cdot \frac{315}{64 \cdot \pi \cdot h^9} \cdot m
\]

\paragraph{Transferencia de la aceleración}
En este paso, para cada bloque de la malla de simulación, se consideran todos 
los bloques contiguos así como el bloque actual. Para cada par de partículas
$i$ y $j$, se evalúan las distancias.

Si las dos partículas están suficientemente cerca 
($\|\vec{p_i} - \vec{p_j}\|^2 < h^2$) se actualiza la aceleración.

\[
\begin{split}
& dist_{ij} = \sqrt{\max(\|\vec{p_i} - \vec{p_j}\|^2, 10^{-12})} \\
& \Delta \vec{a}_{ij} = 
  \frac{
    (\vec{p_i} - \vec{p_j}) \cdot \dfrac{15}{\pi \cdot h^6} \cdot m \cdot 
    \dfrac{(h - dist_{ij})^2}{dist_{ij}} \cdot (\rho_i + \rho_j - 2 \cdot \rho) +
    (\vec{v_j} - \vec{v_i}) \cdot \dfrac{45}{\pi \cdot h^6} \cdot \mu \cdot m
  }{
    \rho_i \cdot \rho_j    
  } \\
& \vec{a}_i = \vec{a}_i + \Delta \vec{a}_{ij}\\
& \vec{a}_j = \vec{a}_j - \Delta \vec{a}_{ij}\\
\end{split}
\]

No obstante, ten especial cuidado para evitar que una partícula se incremente dos
veces con el mismo incremento (ya que $\vec{a}_{ij} = \vec{a}_{ji}$).
