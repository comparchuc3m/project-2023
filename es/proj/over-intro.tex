\subsection{Introducción}

El objetivo de este proyecto es el desarrollo de una aplicación que realice una
simulación de un fluido durante un número de pasos de tiempo,

La aplicación toma los siguientes parámetros de la línea de comandos:

\begin{itemize}
\item \textbf{Iteraciones}: Número de iteraciones o pasos de tiempo a calcular.
\item \textbf{Entrada}: Archivo de datos con la descripción del estado inicial
del fluido.
\item \textbf{Salida}: Archivo de salida con la descripción del estado final del
fluido.
\end{itemize}

Se desarrollará dos versiones de la aplicación para evaluar el impacto de la
estructura de dato seleccionada: estructuras de arrays (\emph{structures of
arrays} o \textmark{SOA}) y arrays de estructuras (\emph{arrays of structures} o
\textmark{AOS}).

\textbad{IMPORTANTE}: Aunque los archivos de entrada y salida contienen
información representada en coma flotante de simple precisión, todos los
cálculos se realizarán utilizando coma flotante en doble precisión.
