\section{Introduction}

Fluid flows simulation can be achieved by multiple methods. One of them is
Smoothed Particle Hydrodynamics (SPH). SPH solves the Navier-Stokes equations
at time steps. A fluid is modeled as a collection of particles. Those
particles have interactions impacting their acceleration, position and velocity.

Only particles at short distance have real interaction. To speedup simulation
time, long distance interactions are discarded. To achieve this, the 3D space
is divided in a grid of blocks and only particles in blocks in the nearby are
considered.

Simulation includes the following steps:
\begin{itemize}
  \item Reading initial state of the simulation.
  \item Simulation of particles for each time step or iteration:
    \begin{itemize}
      \item Repositioning of particles in the grid.
      \item Computing forces and accelerations for each particle.
      \item Processing particle collisions with boundaries.
      \item Particles movement.
      \item Reprocessing box boundaries interactions.
    \end{itemize}
  \item Writing final state of the simulation.
\end{itemize}
