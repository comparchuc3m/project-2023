\section{Introduction}

Fluid flows simulation can be achieved by multiple methods. One of them is
Smoothed Particle Hydrodynamics (SPH). SPH solves the Navier-Stokes equations
at time steps. A fluid is modeled as a collection of particles. Those
particles have interactions impacting their acceleration, position and velocity.

Only particles at short distance have real interaction. To speedup simulation
time, long distance interactions are discarded. To achieve this, the 3D space
is divided in a grid of blocks and only particles in blocks in the nearby are
considered.

Simulation includes the following steps:
\begin{itemize}
\item Computing the density of the fluid at each particle position.
\item Computing the impact of gravity int the acceleration of each particle.
\item Computing fluid pressure.
\item Computing fluid viscosity.
\item Detection of collisions with the walls of the container box.
\item Computing collision response.
\item Determining new position, velocity and acceleration for each particle.
\end{itemize}
