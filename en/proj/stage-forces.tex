\subsubsection{Forces computation}

The following steps are performed

\begin{enumerate}
\item Initialize forces computation.
\item Densities increase.
\item Density transformation.
\item Acceleration transfer.
\end{enumerate}

\paragraph{Initialize forces computation}.
During this step an initial value is given to the density ($\delta_i$) and the
vector acceleration ($\vec{a_i}$) for each particle.

The density is initially set to \cppid{0}.
\[
\delta_i = 0
\]

The vector acceleration is initially set to the \textgood{external acceleration}.

\[
\vec{a} = \vec{g}
\]

\paragraph{Densities increase}
In this step, for every block in the simulation grid all contiguous blocks as
well as the current block are considered. For every pair of particles $i$ and
$j$, the densities are increased.

\[
\Delta \rho_{ij} = 
    \begin{cases}
      (h^2 - |\vec{p_i} - \vec{p_j}|^2)^3 & 
          \text{if  } |\vec{p_i} - \vec{p_j}|^2 < h^2\\
      0 & 
          \text{if  } |\vec{p_i} - \vec{p_j}|^2 \geq h^2\\
    \end{cases}
\]

\paragraph{Density transformation}
For every computed density, a linear transformation is performed.

\[
\rho_i = (\rho_i + h^6) \cdot k_d
\]

\paragraph{Acceleration transfer}
In this step, for every block in the simulation grid all contiguous blocks as
well as the current block are considered. For every pair of particles $i$ and
$j$, distances are evaluated

\[
\begin{split}
\vec{d_{ij}} = \vec{p_i} - \vec{p_j} \\
d_{ij}^2 = | d_{ij} |^2
\end{split}
\]

If the two particles are near enough ($d_{ij}^2 < h^2$) acceleration is updated.

\[
\begin{split}
& dist_{ij} = \sqrt{\max(d_{ij}^2, 10^{-12})} \\
& \Delta \vec{a} = \vec{d_{ij}} \cdot k_p \cdot \frac{(h - dist_{ij})^2}{dist_{ij}^2} \cdot
(d_i + d_j - \rho) \\
&a_i = a_i + \Delta \vec{a}\\
&a_j = a_j - \Delta \vec{a}
\end{split}
\]
