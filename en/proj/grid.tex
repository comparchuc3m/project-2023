\subsection{The simulation grid}

To avoid excessive computations the whole simulation box is divided in smaller
blocks. Note that the simulation box is a rectangular cuboid with different
lengths in each of the three dimensions. 

To determine the number of blocks in each dimension the length int that
dimensions is divided by parameter $h$.

\[
\begin{split}
n_x = \lfloor \frac{x_{max}- x_{min}}{h} \rfloor\\
n_y = \lfloor \frac{y_{max}- y_{min}}{h} \rfloor\\
n_z = \lfloor \frac{z_{max}- z_{min}}{h} \rfloor
\end{split}
\]

The size of grid block $\vec{s}$ ($s_x, s_y, s_z$) is then given by
\[
\begin{split}
s_x = \frac{x_{max} - x_{min}}{n_x}\\
s_y = \frac{y_{max} - y_{min}}{n_y}\\
s_z = \frac{z_{max} - z_{min}}{n_z}\\
\end{split}
\] 

Given a particle with position vector $\vec{p}$ the block indices are computed
by considering its distance in each dimension to the box minimum coordinate
and the size of a grid block ($\vec{s}$).

\[
\begin{split}
i = \frac{p_x - x_{min}}{s_x}\\
j = \frac{p_y - y_{min}}{s_y}\\
k = \frac{p_z - z_{min}}{s_z}\\
\end{split}
\]

Note that for each dimensions the corresponding index value cannot be less than
$0$ or greater or equal than the corresponding number of blocks in that
dimension:

\[
\begin{split}
i \in [0, n_x-1]\\
j \in [0, n_x-1]\\
k \in [0, n_x-1]\\
\end{split}
\]

If an index value is outside its corresponding range, it shall be adjusted to
the corresponding limit.

The key role of simulation blocks is that only particles in the same block or
contiguous blocks will be computed. In general, for any given block there are
26 contiguous blocks. This number may be lower when the block is in the limits
of the grid. For example, when $i=j=k=0$, the number of contiguous blocks is 7.
