\subsection{Simulation initialization}

During initialization the input file is opened and the main simulation
parameters are read (particles per meter and number of particles).
Then, other parameters depending on them are computed.

Those parameters include:
\begin{itemize}
\item The \textmark{smoothing length} ($h$).
\item The \textmark{particle mass} ($m$).
\item The \textmark{grid size vector} ($\vec{n} = (n_x, n_y, n_z)$).
\item The \textmark{grid block size} ($\vec{s} = (s_x, s_y, s_z)$).
\end{itemize}

The information for each particle is read from the input file. Although input
files use single precision for floating point as soon as a value is read it
shall be converted to double precision.  Particles will be assigned an integer
numeric identifier starting at \cppid{0}.

Once a particle has been read, its corresponding block in the grid determined
and the particle is added to the set of particles in that block. 

If the number of particles found in the file does not match with the parameter
\textmark{number of particles} (\cppid{np}), an error message shall be generated
and the program shall exit with error code \cppid{-5}. For example, if the 
header has the value \cppid{4750} for \cppid{np} but the file contains 
\cppid{4700} particles, the following message shall be sent to the standard
error output.

\begin{lstlisting}[style=terminal,escapechar=@]
@\$@ aos 2000 init.fld final.fld
Error: Number of particles mismatch. Header: 4750, Found: 4700.
\end{lstlisting}

If the file is successfully read the values for the parameters shall be sent
to the standard output.

\begin{lstlisting}[style=terminal,escapechar=@]
@\$@ aos 2000 init.fld final.fld
Number of particles: 4800
Particles per meter: 204
Smoothing length: 0.00830882
Particle mass: 0.00011779
Grid size: 15 x 21 x 15
Number of blocks: 4725
Block size: 0.00866667 x 0.00857143 x 0.00866667
\end{lstlisting}

