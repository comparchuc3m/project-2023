\subsection{Simulation initialization}

During initialization the input file is opened and the main simulation
parameters are read (particles per meter and number of particles).
Then, other parameters depending on them are computed.

Those parameters include:
\begin{itemize}
\item The \textmark{smoothing length} ($h$).
\item The \textmark{particle mass} ($m$).
\item The \textmark{grid size vector} ($\vec{n} = (n_x, n_y, n_z)$).
\item The \textmark{grid block size} ($\vec{s} = (s_x, s_y, s_z)$).
\end{itemize}

The information for each particle is read from the input file. Although
input files use single precision for floating point as soon as a value is 
read it shall be converted to double precision.
Particles will be assigned an integer numeric identifier starting at \cppid{0}.

Once a particle has been read, its corresponding block in the grid determined
and the particle is added to the set of particles in that block. 
