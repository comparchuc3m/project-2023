\subsection{Program arguments parsing}

Each of the applications (\cppid{soaf}, \cppid{soad}, \cppid{aosf} and
\cppid{aosd}) will be given exactly three paramters:

\begin{itemize}
\item \textmark{nts}: Number of time steps. This parameter specifies how many
time steps shall be the simulation executed.
\item \textmark{inputfile}: Input file with initial simulation state.
\item \textmark{outputfile}: Output file where results shall be written.
\end{itemize}

For example, the command:

\begin{lstlisting}[style=terminal,escapechar=@]
@\$@ aosf 2000 init.fld final.fld
\end{lstlisting}

Loads the file \cppid{init.fld} runs \cppid{2000} time steps and generates an
output file named \cppid{final.fld}.

If number of time steps is not a positive number an error message shall be
generated and the program shall exit with error code \cppid{-1}.

\begin{lstlisting}[style=terminal,escapechar=@]
@\$@ aosf -3 init.fld final.fld
Error: Invalid number of time steps
\end{lstlisting}

If the input file cannot be opened for reading an error message shall be
generated and the program shall exit with error code \cppid{-2}.

\begin{lstlisting}[style=terminal,escapechar=@]
@\$@ soad 1 init.fld final.fld
Error: Cannot open init.fld for reading
\end{lstlisting}

If the otput file cannot be opened for writing an error message shall be
generated and the program shall exit with error code \cppid{-3}.

\begin{lstlisting}[style=terminal,escapechar=@]
@\$@ soad f init.fld final.fld
Error: Cannot open final.fld for writing
\end{lstlisting}
