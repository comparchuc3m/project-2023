\subsection{File format}

A fluid file contains all the information of fluid state at a given point in
time. Files are stored in binary format with the following requirements:

\begin{itemize}
\item \textmark{Integer}: All integer values use 4 bytes.
\item \textmark{Floating-point}: All floating point values use IEEE-754 single
precision representation.
\end{itemize}

A file is composed of a \textgood{header} and a \textgood{body}. The header
includes general information while the body contains information for each
particle in the fluid.

The header contains the fields as specified in Table~\ref{tab:file-header}.

\begin{table}[htbp]
\begin{center}
\begin{tabular}{|l|l|l|}
\hline
Field & Type & Description\\
\hline
\hline

\textgood{ppm} & \textemph{Floating-Point} &
Particles per meter\\
\hline

\textgood{np} & \textemph{Integer} &
Number of particles\\
\hline
\end{tabular}
\end{center}
\label{tab:file-header}
\caption{Fields in the file header.}
\end{table}

Note that the \textmark{number of particles} specifies how many particles are
stored in the rest of the file. The information and format for each particle in
the file \textgood{body} is specified in Table~\ref{tab:file-body}.

\begin{table}[htbp]
\begin{center}
\begin{tabular}{|l|l|l|}
\hline
Field & Type & Description\\
\hline
\hline

\textgood{px} & \textemph{Floating-Point} &
Position x-coordinate.\\
\hline
\textgood{py} & \textemph{Floating-Point} &
Position y-coordinate.\\
\hline
\textgood{pz} & \textemph{Floating-Point} &
Position z-coordinate.\\
\hline

\textgood{hvx} & \textemph{Floating-Point} &
Vector hv x-coordinate.\\
\hline
\textgood{hvy} & \textemph{Floating-Point} &
Vector hv y-coordinate.\\
\hline
\textgood{hvz} & \textemph{Floating-Point} &
Vector hv z-coordinate.\\
\hline

\textgood{vx} & \textemph{Floating-Point} &
Acceleration x-coordinate.\\
\hline
\textgood{vy} & \textemph{Floating-Point} &
Acceleration y-coordinate.\\
\hline
\textgood{vz} & \textemph{Floating-Point} &
Acceleration z-coordinate.\\
\hline

\hline
\end{tabular}
\end{center}
\label{tab:file-body}
\caption{Fields in the file body for each particle.}
\end{table}
