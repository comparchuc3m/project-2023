\clearpage
\subsection{Simulation constants}

Table~\ref{tab:constants} shows the constants used in the simulation.

\begin{table}[hbt!]

\begin{center}
\begin{tabular}{|l|l|r|}

\hline
\textbf{Constant} & \textbf{Description} & \textbf{Value}\\
\hline
\hline

r & \textgood{Radius multiplier} & 1.695\\
\hline

$\rho$ & \textgood{Fluid density} & $10^3$\\
\hline

$p_s$ & \textgood{Stiffness pressure} & 3.0\\
\hline 

$s_c$ & \textgood{Stiffness collisions} & $3 \cdot 10^4$\\
\hline

$d_v$ & \textgood{Damping} & 128.0\\
\hline

$\mu$ & \textgood{Viscosity} & 0.4\\
\hline

$d_p$ & \textgood{Particle size} & $2 \cdot 10^{-4}$\\
\hline

$\Delta t $& \textgood{Time step} & $10^{-3}$\\
\hline

\end{tabular}
\end{center}

\caption{Simulation scalar constants.}
\label{tab:constants}
\end{table}

In addition, Table~\ref{tab:vec-constants} shows vector constants to be used in the simulation.
Note that the \textgood{external acceleration} ($\vec{g}$) is the gravity acceleration which in
this example acts across the y-axis. The \textgood{box upper bound} ($\vec{b}_{max}$) 
and \textgood{box lower bound} ($\vec{b}_{min}$) define two opposite vertex in a rectangular
cuboid. This is the closed domain where the simulation happens. $\vec{b}_{max}$ is given
by vector $(x_{max}, y_{max}, z_{max})$ and $\vec{b}_{min}$ is given by vector
$(x_{min}, y_{min}, z_{min})$.

\begin{table}[h]

\begin{center}
\begin{tabular}{|l|l|r|r|r|}
  
\hline
\textbf{Constant} & \textbf{Description} & \cppid{x} & \cppid{y} & \cppid{z}\\
\hline
\hline

$\vec{g}$ & \textgood{External acceleration} & 0.0 & -9.8 & 0.0\\
\hline

$\vec{b}_{max}$ & \textgood{Box upper bound} & 
\cppid{$x_{max}$} = 0.065 & \cppid{$y_{max}$} = 0.1 & \cppid{$z_{max}$} = 0.065\\
\hline

$\vec{b}_{min}$ & \textgood{Box lower bound} & 
\cppid{$x_{min}$} = -0.065 & \cppid{$y_{min}$} = -0.08 & \cppid{$z_{min}$} = -0.065\\
\hline 

\end{tabular}
\end{center}

\caption{Simulation vector constants.}
\label{tab:vec-constants}
\end{table}

Figure~\ref{fig:cube} shows a possible grid divided in blocks and box bounds
$b_{min}$ and $b_{max}$.

\begin{figure}[htb!]
\begin{center}
\input{int/cube.tkz}
\end{center}
\caption{Cube dimension.}
\label{fig:cube}
\end{figure}

Note that some mathematical constants (e.g. $\pi$) may be needed. In that 
case the preferred value shall be the one provided by the C++20 standard 
library. Please, refer to the namespace \cppid{std::numbers} for such constants
(see \url{https://en.cppreference.com/w/cpp/header/numbers}).
